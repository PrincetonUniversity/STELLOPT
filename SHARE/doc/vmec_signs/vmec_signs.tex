\documentclass[11pt,letter]{article}

\usepackage{latexsym, color, graphicx, comment}
\usepackage[top=1in,bottom=1in,left=1in,right=1in]{geometry}
\usepackage{amssymb}
\usepackage{amsmath}
\usepackage{hyperref}
\hypersetup{colorlinks=true,linkcolor=blue}


%\graphicspath{{../MATLAB/}}

\newcommand{\vect}[1]{\mbox{\boldmath $#1$}}
\DeclareMathAlphabet{\mathbfsf}{\encodingdefault}{\sfdefault}{bx}{n}
\newcommand{\tens}[1]{\mathbfsf{#1}}

\newcommand{\red}[1]{\textcolor{red}{#1}}
\newcommand{\erf}{\mathrm{erf}}
\newcommand{\PS}{Pfirsch-Schl\"{u}ter~}
\newcommand{\ExB}{\vect{E}\times\vect{B}}
\newcommand{\Cnl}{C_{\mathrm{nl}a}}
\newcommand{\vma}{\vect{v}_{\mathrm{m}a}}
\newcommand{\vm}{\vect{v}_{\mathrm{m}}}
\newcommand{\vda}{\vect{v}_{\mathrm{d}a}}
\newcommand{\vE}{\vect{v}_{E}}
\newcommand{\vdao}{\vect{v}_{\mathrm{d}a0}}
\newcommand{\vdo}{\vect{v}_{\mathrm{d}0}}
\newcommand{\vEo}{\vect{v}_{E0}}
\newcommand{\vpar}{v_{||}}
\newcommand{\vperp}{v_{\bot}}
\newcommand{\rhop}{\rho_{\theta}}
\newcommand{\rT}{r_{T}}
\newcommand{\Ti}{T_{\mathrm{i}}}
\newcommand{\rTi}{r_{T\mathrm{i}}}
\newcommand{\rTe}{r_{T\mathrm{e}}}
\newcommand{\reta}{r_{\eta}}
\newcommand{\rn}{r_{n}}
\newcommand{\rperp}{r_\bot}
\newcommand{\fM}{f_{\mathrm{M}}}
\newcommand{\fMa}{f_{\mathrm{M}a}}
\newcommand{\fMi}{f_{\mathrm{M}i}}
\newcommand{\fMe}{f_{\mathrm{Me}}}
\newcommand{\fe}{f_{\mathrm{e}}}
\newcommand{\Te}{T_{\mathrm{e}}}
\newcommand{\vth}{v_{\mathrm{th}}}
\newcommand{\nuii}{\nu_{ii}}
\newcommand{\nuPrime}{\hat{\nu}}
\newcommand{\gradpar}{\nabla_{||}}
\newcommand{\kpar}{k_{||}}
\newcommand{\psia}{\psi_0}
\newcommand{\psiaHat}{\hat{\psi}_0}
\newcommand{\omegase}{\omega_{*}}
\newcommand{\nuee}{\nu_{ee}}
\newcommand{\energy}{\mathcal{E}}
\newcommand{\sgn}{\mathrm{sgn}}
\newcommand{\changed}[1]{\textcolor{red}{#1}}
\newcommand{\todo}[1]{\textcolor{green}{#1}}
\newcommand{\nescoil}{{\tt NESCOIL}}
\newcommand{\regcoil}{{\tt REGCOIL}}
\newcommand{\vmec}{{\tt VMEC}}
\newcommand{\makegrid}{{\tt MAKEGRID}}
\newcommand{\currentPot}{\Phi}
\newcommand{\currentPotSV}{\Phi_{sv}}
\newcommand{\Bnormal}{B_{\mathrm{normal}}}
\newcommand{\sincos}{\begin{pmatrix} \sin \\ \cos \end{pmatrix}}
\newcommand{\cossin}{\begin{pmatrix} \cos \\ -\sin \end{pmatrix}}
\newcommand{\Bnplasma}{B_{\mathrm{normal}}^{\mathrm{plasma}}}
\newcommand{\Bnexternal}{B_{\mathrm{normal}}^{\mathrm{external}}}
\newcommand{\Bncurrentpot}{B_{\mathrm{normal}}^{\Phi}}
\newcommand{\BnormalSV}{B_{\mathrm{normal}}^{\mathrm{sv}}}

\begin{document}

\title{Sign conventions in VMEC}


% repeat the \author .. \affiliation  etc. as needed
% \email, \thanks, \homepage, \altaffiliation all apply to the current author.
% Explanatory text should go in the []'s,
% actual e-mail address or url should go in the {}'s for \email and \homepage.
% Please use the appropriate macro for the type of information

% \affiliation command applies to all authors since the last \affiliation command.
% The \affiliation command should follow the other information.

%\author{Pavlos Xanthopoulos}
%\email[]{mattland@umd.edu}
%\homepage[]{Your web page}
%\thanks{}
%\affiliation{Max Planck Institute for Plasma Physics, Greifswald, Germany}

\author{Matt Landreman}
%\email[]{mattland@umd.edu}
%\homepage[]{Your web page}
%\thanks{}
%\affiliation{Institute for Research in Electronics and Applied Physics, University of Maryland, College Park, MD, 20742, USA}

% Collaboration name, if desired (requires use of superscriptaddress option in \documentclass).
% \noaffiliation is required (may also be used with the \author command).
%\collaboration{}
%\noaffiliation

\date{September 28, 2017}
\maketitle %\maketitle must follow title, authors, abstract and \pacs


%\section{Summary of main points}

This note is based on some preliminary numerical experiments, and I have not yet been able to confirm them by
examination of the source code, so there could well be errors.

The following points apply to the case {\ttfamily lrfp=.false.}; I am not sure if anything is different in the {\ttfamily lrfp=.true.} case.

\begin{enumerate}
\item
The toroidal angle, variously called $\phi$, $\zeta$, or $v$, always increases in a counter-clockwise direction when the plasma is viewed from above.
Hence, the cylindrical coordinate system $(R,\zeta,Z)$ is right-handed.
(I've verified this is true for \makegrid, and hence true for free-boundary \vmec, but I'm not sure it needs to be true for fixed-boundary \vmec.)
\item
The direction of the poloidal angle, variously called $\theta$ or $u$, always increases as you move from the outboard to inboard
side over the top of the plasma. If the boundary shape specified in the input file implies the opposite direction for $\theta$,
vmec will flip the sign of some of the shape coefficients to flip the sign of $\theta$. As a result, the Jacobian of the $(s,\theta,\zeta)$
coordinates 
\begin{equation}
\sqrt{g} = \frac{\partial \vect{r}}{\partial s} \cdot \frac{\partial\vect{r}}{\partial\theta} \times \frac{\partial\vect{r}}{\partial\zeta}
= \frac{1} {\nabla s \cdot \nabla\theta\times\nabla\zeta}
\end{equation}
is always negative. Here, $\vect{r}$ is the position vector, and $s$ is the toroidal flux normalized to its value at the edge, so $s=0$ on the magnetic axis and $s=1$ at the outermost surface.
This sign of $\sqrt{g}$ is reflected in the output variable {\ttfamily signgs}, which is always $-1$.
(Then why is {\ttfamily signgs} saved, if it is always the same?)
\item
These previous points imply that the rotational transform $\iota = d(\theta + \lambda) / d\zeta$ is positive if the magnetic
field lines spiral with a left-handed orientation, whereas $\iota<0$ if the field lines make a right-handed spiral.
\item
A positive value for the input toroidal flux variable {\ttfamily phiedge} corresponds to the magnetic field pointing in the $\nabla \zeta$ direction,
(counter-clockwise when viewed from above.) This flux satisfies
\begin{equation}
\mathtt{phiedge} = \int_0^s ds' \int_0^{2\pi} d\theta | \sqrt{g}| \vect{B}\cdot\nabla\zeta,
\end{equation}
where the absolute value must be included around the $| \sqrt{g}|$ factor.
\item
The output toroidal flux quantities {\ttfamily phi} and {\ttfamily phipf} have the same sign as {\ttfamily phiedge}.
However, the output toroidal flux quantity {\ttfamily phips} differs in sign by {\ttfamily signgs}, i.e. it has the opposite sign.
\item
The toroidal plasma current given by the input parameter {\ttfamily curtor} and output parameter {\ttfamily ctor} is positive if the current points in the $\nabla\zeta$ direction
(counter-clockwise when viewed from above.) This current satisfies
\begin{equation}
\mathtt{curtor} = \int_0^s ds' \int_0^{2\pi} d\theta | \sqrt{g}| \vect{j}\cdot\nabla\zeta,
\end{equation}
where $\vect{j}$ is the current density, and the absolute value must be included around the $| \sqrt{g}|$ factor.
\item
In the expression for the magnetic field
\begin{equation}
\vect{B} = \nabla\psi\times\nabla(\theta + \lambda) + \iota \nabla\zeta\times\nabla\psi,
\end{equation}
the sign of the $\psi$, which is the toroidal flux divided by $2\pi$, differs from the sign of vmec's {\ttfamily phi}$/(2\pi)$
by {\ttfamily signgs}$=-1$.
\end{enumerate}

\begin{comment}
\section{Evidence}
Line 273 of {\ttfamily General/bcovar.f: \\
\\
ctor = signgs*twopi*(c1p5*buco(ns) - p5*buco(ns1))
\\
}
where {\ttfamily p5}=0.5 and {\ttfamily c1p5}=1.5.
\end{comment}

%\appendix

%\bibliography{quasisymmetricEquilibria}


\end{document}



