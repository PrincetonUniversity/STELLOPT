\documentclass[11pt,letter]{article}

\usepackage{latexsym, color, graphicx, comment}
\usepackage[top=1in,bottom=1in,left=1in,right=1in]{geometry}
\usepackage{amssymb}
\usepackage{amsmath}
\usepackage{hyperref}
\usepackage{empheq}
\hypersetup{colorlinks=true,linkcolor=blue}

\newcommand{\vect}[1]{\mbox{\boldmath $#1$}}
\newcommand{\gyrophase}{\varphi}
\newcommand{\energy}{\varepsilon}
\renewcommand{\Re}{\mathrm{Re}}
\renewcommand{\Im}{\mathrm{Im}}
\newcommand{\todo}[1]{\textcolor{red}{#1}}
%\newcommand{\kxfac}{\kappa_x}
\newcommand{\kxfac}{\mathtt{kxfac}}
\newcommand{\bmag}{\mathtt{bmag}}
\newcommand{\smz}{\mathtt{smz}}
\newcommand{\gdstwo}{\mathtt{gds2}}
\newcommand{\gdstwoone}{\mathtt{gds21}}
\newcommand{\gdstwotwo}{\mathtt{gds22}}
\newcommand{\gbdrift}{\mathtt{gbdrift}}
\newcommand{\gbdriftO}{\mathtt{gbdrift0}}
\newcommand{\cvdrift}{\mathtt{cvdrift}}
\newcommand{\cvdriftO}{\mathtt{cvdrift0}}
\newcommand{\fprim}{\mathtt{fprim}}
\newcommand{\tprim}{\mathtt{tprim}}
\newcommand{\gradpar}{\mathtt{gradpar}}
\newcommand{\delthet}{\mathtt{delthet}}
\newcommand{\codedt}{\mathtt{code\_dt}}
\newcommand{\stellopt}{{\ttfamily stellopt}}
\newcommand{\vboot}{{\ttfamily vboot}}
\newcommand{\vmec}{{\ttfamily vmec}}
\newcommand{\bootsj}{{\ttfamily bootsj}}
\newcommand{\sfincs}{{\ttfamily sfincs}}
\newcommand{\curtor}{{\ttfamily curtor}}
\newcommand{\ac}{{\ttfamily ac}}


\title{Computing \vmec's \ac~current profile and \curtor~from a bootstrap current code}
\author{Matt Landreman}
\date{March 1, 2018}

\begin{document}
\maketitle

In this note, we detail how to compute the \vmec~input parameters \curtor~and \ac~(or {\ttfamily ac\_aux\_f})
from the averaged parallel current computed by neoclassical codes, $\langle \vect{j}\cdot\vect{B} \rangle$. Here,
$\langle \ldots \rangle$ denotes a flux surface average,  $\vect{j}$ is the current density, and $\vect{B}$ is the magnetic field. 
The proper result involves a term that is neglected by the \bootsj~code. This term is small
in $\beta = 2 \mu_0 p / B^2$, but it does not require much work to keep, so we may as well keep it.
Physically, neoclassical codes compute a gyrophase-averaged distribution function and so the current they obtain
is parallel to $\vect{B}$, whereas the total toroidal current also has a contribution from the diamagnetic
current perpendicular to $\vect{B}$. The purpose of this note is to compute the proper relationship between these 
parallel and perpendicular currents and the net toroidal current.


The calculation here is a more detailed version of Appendix C of \cite{LandremanCatto},
except that Gaussian units were used in that paper whereas
SI units are used throughout this note. 


\section{Overview}
Bootstrap current codes like \bootsj~and \sfincs~solve the drift kinetic equation
for the gyrophase-averaged distribution function $f$. The parallel velocity moment of $f$, weighted by species charge and summed over
species, yields the parallel current $j_{||} = \vect{j}\cdot\vect{b}$, where $\vect{b} = \vect{B} / |\vect{B}|$.
The spatial variation of $j_{||}$ over a magnetic surface can be determined analytically for any geometry and collisionality using mass conservation, so the results of a drift-kinetic calculation can be summarized by any weighted average of the parallel current over a flux surface. By convention, the average typically
reported is $\langle \vect{j}\cdot\vect{B}\rangle$.

The \vmec~code takes as an input the profile of net toroidal current inside a flux surface.
Therefore, to interface a bootstrap current code with \vmec, the relationship between this
net toroidal current and the parallel current must be calculated. We now calculate this relationship.

\section{Derivation}

Let $I(s)$ denote the total toroidal current inside a flux surface labelled by any flux function $s$,
where we require that $s=0$ on the magnetic axis.
This total current is the flux (area integral) of the current density $\vect{j}$ through a surface of constant toroidal angle $\zeta$:
\begin{equation}
I(s) = \int d^2\vect{a} \cdot \vect{j}
=\int_0^{s} ds' \int_0^{2\pi}d\theta \sqrt{g} \vect{j}\cdot\nabla\zeta,
\label{eq:I_definition}
\end{equation}
where $\theta$ is a poloidal angle, the integrand is evaluated at $s'$ rather than $s$, and
\begin{equation}
\sqrt{g} = \frac{\partial \vect{r}}{\partial s} \cdot \frac{\partial\vect{r}}{\partial\theta} \times \frac{\partial\vect{r}}{\partial\zeta}
= \frac{1} {\nabla s \cdot \nabla\theta\times\nabla\zeta}
\end{equation}
is the Jacobian of the $(s,\theta,\zeta)$ coordinates. 
It turns out to be convenient to write (\ref{eq:I_definition}) in differential rather than integral form, by applying $d/ds$:
\begin{equation}
\frac{dI}{ds}
= \int_0^{2\pi}d\theta \sqrt{g} \vect{j}\cdot\nabla\zeta.
\label{eq:I_differential}
\end{equation}
These expressions so far are all valid for any angle coordinates $(\theta,\zeta)$ (\vmec, Boozer, {\ttfamily pest}, Hamada, etc.)

Next, consider that to leading order in $\rho/L \ll 1$, where $\rho$ is a typical gyroradius and $L$ is a typical equilibrium scale length,
the current perpendicular to $\vect{B}$ is given by the diamagnetic current:
\begin{equation}
\vect{j}_\perp = \frac{1}{B^2} \frac{dp}{ds} \vect{B}\times\nabla s,
\end{equation}
where $p(s)$ is the total pressure.
This result can be obtained, for example, by applying $\vect{B}\times(\ldots)$ to the MHD equilibrium relation 
$\vect{j}\times\vect{B} = \nabla p$. Therefore the total current vector is
\begin{equation}
\vect{j} = \frac{j_{||}}{B} \vect{B} +  \frac{1}{B^2} \frac{dp}{ds} \vect{B}\times\nabla s.
\end{equation}
Substituting this result into (\ref{eq:I_differential}),
\begin{equation}
\frac{dI}{ds}=
\int_0^{2\pi}d\theta \sqrt{g} 
\left[
\frac{j_{||}}{B} \vect{B} \cdot\nabla\zeta +  \frac{B_\theta}{B^2} \frac{dp}{ds} \nabla\theta \times\nabla s\cdot\nabla\zeta
\right],
\label{eq:I2}
\end{equation}
where to get the last term we have expressed $\vect{B}$ in components
\begin{equation}
\vect{B} = B_s \nabla s + B_\theta \nabla \theta + B_\zeta \nabla\zeta.
\label{eq:B_components}
\end{equation}
The important message from (\ref{eq:I2}) is that the total toroidal current $I$ needed for \vmec~consists
 not only of the parallel current determined by neoclassical physics (the first right-hand-side term), but
also by the diamagnetic current $\propto dp/ds$ in the last term.

While the calculation so far is true for any angle coordinates $(\theta,\zeta)$,
for the rest of this section it is convenient to use Boozer coordinates.
The main result will turn out to be independent of the choice of angle coordinates.
In any straight-field-line coordinates such as Boozer coordinates, the magnetic field can be written as
\begin{equation}
\vect{B} = \frac{d\psi}{ds} \left[ \nabla s \times\nabla \theta + \iota \nabla \zeta\times\nabla s\right],
\label{eq:straight_field_lines}
\end{equation}
where $2\pi\psi$ is the toroidal flux enclosed by surface $s$, and $\iota(s)$ is the rotational transform. Hence, the geometric factor appearing
in the first right-hand-side term of (\ref{eq:I2}) is
\begin{equation}
\vect{B}\cdot\nabla\zeta =  \frac{d\psi}{ds}  \nabla s \times\nabla \theta \cdot\nabla\zeta = \frac{d\psi}{ds} \frac{1}{\sqrt{g}},
\end{equation}
and so (\ref{eq:I2}) reduces to
\begin{equation}
\frac{dI}{ds} =
 \int_0^{2\pi}d\theta 
\left[
\frac{d\psi}{ds}
\frac{j_{||}}{B}  -  \frac{B_\theta}{B^2} \frac{dp}{ds} 
\right].
\label{eq:I3}
\end{equation}

Furthermore, in Boozer coordinates, $B_\theta$ and $B_\zeta$ in (\ref{eq:B_components})
are flux functions, i.e. they depend only on $s$,
and we can show that $B_\theta$ is related to $I(s)$ as follows.
The curl of (\ref{eq:B_components}) is
\begin{equation}
\nabla\times\vect{B} =\nabla B_s \times \nabla s + \frac{d B_\theta}{ds} \nabla s \times \nabla \theta + \frac{d B_\zeta}{ds} \nabla s \times \nabla\zeta.
\end{equation}
Substituting this expression into Ampere's Law $\mu_0 \vect{j} = \nabla\times\vect{B}$,
and using the result to eliminate $\vect{j}$ in 
(\ref{eq:I_definition}),
\begin{align}
\mu_0 I(s) &= \int_0^{s} ds' \int_0^{2\pi}d\theta \sqrt{g} 
\left[ 
\frac{\partial B_s}{\partial\theta}
\nabla \theta \times \nabla s \cdot\nabla\zeta + \frac{d B_\theta}{ds} \nabla s \times \nabla \theta \cdot\nabla\zeta
\right] \\
&= \int_0^{s} ds' \int_0^{2\pi}d\theta 
\left[ -
\frac{\partial B_s}{\partial\theta}
+ \frac{d B_\theta}{ds} 
\right] = \int_0^{s} ds' \int_0^{2\pi}d\theta 
 \frac{d B_\theta}{ds}  \nonumber \\
 &=2\pi [ B_\theta(s) - B_\theta(0) ]. \nonumber
\end{align}
At the magnetic axis ($s=0$), $B_\theta$ must vanish, since in (\ref{eq:B_components}) it multiplies $\nabla\theta$ which diverges on axis, and the product
must be regular. Hence,
\begin{equation}
B_\theta (s) = \frac{\mu_0 I(s)}{2\pi}.
\end{equation}
Using this result in (\ref{eq:I3}), and applying a toroidal average $(2\pi)^{-1} \int_0^{2\pi} d\zeta (\ldots)$,
\begin{equation}
 \frac{dI}{ds} =
 \frac{1}{2\pi}
\frac{d\psi}{ds}
\int_0^{2\pi}d\theta  \int_0^{2\pi} d\zeta \frac{j_{||}}{B}
-
\frac{\mu_0 I}{4\pi^2}
\frac{dp}{ds} 
\int_0^{2\pi}d\theta  \int_0^{2\pi} d\zeta
 \frac{1}{B^2} .
\label{eq:I4}
\end{equation}

The angular averages in this last expression can be written in terms of the flux surface average, which for any quantity $Q$
is
\begin{equation}
\langle Q \rangle
= \frac{ \int_0^{2\pi}d\theta \int_0^{2\pi} d\zeta \sqrt{g} Q}{ \int_0^{2\pi}d\theta \int_0^{2\pi} d\zeta \sqrt{g}}
=\frac{ \int_0^{2\pi}d\theta \int_0^{2\pi} d\zeta \sqrt{g} (Q/B^2)}{ \int_0^{2\pi}d\theta \int_0^{2\pi} d\zeta /B^2}.
\label{eq:FSA}
\end{equation}
In the last equation, we have used the fact that the Jacobian in Boozer coordinates is $\sqrt{g} = (d\psi / ds) (B_\zeta + \iota B_\theta) / B^2$,
which follows from the product of (\ref{eq:B_components}) and (\ref{eq:straight_field_lines}).
From (\ref{eq:FSA}) we see that
\begin{equation}
\langle B^2 \rangle = \frac{ 4 \pi^2}{ \int_0^{2\pi}d\theta \int_0^{2\pi} d\zeta /B^2}.
\label{eq:FSAB2}
\end{equation}
Using (\ref{eq:FSA})-(\ref{eq:FSAB2}), (\ref{eq:I4}) can be written
\begin{equation}
\boxed{
 \frac{dI}{ds} 
 +\frac{\mu_0 I}{\langle B^2 \rangle}
\frac{dp}{ds} 
 =
2\pi
\frac{d\psi}{ds}
\frac{ \langle j_{||} B \rangle} {\langle B^2 \rangle}.
}
\label{eq:I5}
\end{equation}
This result is the key equation for relating
the current density $\langle j_{||} B \rangle$ from a neoclassical code to the radial profile 
of total current in an MHD equilibrium code.
While we used Boozer angles to derive this result,% main result (\ref{eq:main_result}),
%all the expressions from (\ref{eq:I5})-(\ref{eq:main_result}) are 
it is independent of any particular choice of angles.

%Since $I$ appears both with and without a radial derivative in (\ref{eq:I5}), some thought needs to be given to
There are several ways to implement (\ref{eq:I5}) numerically, which we will now describe.
In each case, an iteration must be performed between the MHD equilibrium code and the bootstrap current code.
We will use a subscript $i$ to denote the iteration step.

\subsection{Low $\beta$ approximation}
It can be seen that the $dp/ds$ term in (\ref{eq:I5}) is smaller than the $dI/ds$ term preceding it by a factor of $\beta$.
Therefore for $\beta\ll 1$ it is a reasonable approximation to neglect the $dp/ds$ term. Then
the current profile is updated according to
\begin{equation}
 \frac{dI_{i+1}}{ds} 
 =
2\pi
\frac{d\psi}{ds}
\frac{ \langle j_{||} B \rangle_i} {\langle B_i^2 \rangle}.
\label{eq:iteration_low_beta}
\end{equation}
This is the approach adopted in \bootsj.

\subsection{Lagging the $dp/ds$ term}
A more accurate approach to solving (\ref{eq:I5}) numerically is to evaluate the updated $dI/ds$ term using the $dp/ds$ term
from the previous iteration:
\begin{equation}
 \frac{dI_{i+1}}{ds} 
 =-\frac{\mu_0 I_i}{\langle B_{i}^2 \rangle}
\frac{dp}{ds} 
+
2\pi
\frac{d\psi}{ds}
\frac{ \langle j_{||} B \rangle_{i}} {\langle B_{i}^2 \rangle}.
\label{eq:iteration_lag}
\end{equation}
 For $\beta\ll1$ this iteration should converge rapidly since the $dp/ds$ term is small, and since the factor of $I$ in the $dp/ds$ term is smoother than $dI/ds$. This is the approach that is used for the \stellopt~\vboot~iteration using \sfincs.

\subsection{Integrating factor}
A third approach to solving (\ref{eq:I5}) is to interpret this expression as an ordinary differential equation for $I(s)$, introducing an integrating factor:
\begin{equation}
\frac{d}{ds} \left( I F\right)  =
2\pi F
\frac{d\psi}{ds}
\frac{ \langle j_{||} B \rangle} {\langle B^2 \rangle},
\label{eq:main_result_differential}
\end{equation}
where the integrating factor is
\begin{equation}
F(s) =   \exp\left( \mu_0 \int_0^s \frac{ds'}{\langle B^2 \rangle} \frac{dp}{ds} \right).
\end{equation}
(Again, the integrand is evaluated at $s'$ rather than $s$.)
One can see from this definition that $F = 1 + \mathcal{O}(\beta)$.
Imposing the boundary condition $I(0)=0$
(there is no enclosed toroidal current on the magnetic axis),
the solution to (\ref{eq:main_result_differential}) is then
\begin{equation}
 I(s)  = \frac{2\pi}{F(s)} \int_0^s ds''
F
\frac{d\psi}{ds}
\frac{ \langle j_{||} B \rangle} {\langle B^2 \rangle},
\label{eq:main_result}
\end{equation}
where the integrand is evaluated at $s''$. From this expression,
we obtain the iterative scheme
\begin{equation}
 I_{i+1}(s)  = \frac{2\pi}{F_i(s)} \int_0^s ds''
F_i
\frac{d\psi}{ds}
\frac{ \langle j_{||} B \rangle_i} {\langle B_i^2 \rangle},
\label{eq:iteration_integrating_factor}
\end{equation}

\section{VMEC definitions}



The current profile is provided to \vmec~using two quantities. The first is the number \curtor, which is
the total toroidal current inside the outermost \vmec~magnetic surface. My understanding of \vmec's sign convention is
that \curtor~is positive when the current is in the $\nabla\zeta$ direction, where $\zeta$ is the toroidal angle
used in \vmec~(also called $\phi$ or $v$ in the code and its documentation).
I also believe \vmec's $\zeta$ always increases in the counter-clockwise direction when the plasma
is viewed from above, so $(R,\zeta,Z)$ is a right-handed system.
For comparison, (\ref{eq:I_definition}) indicates that $I(s)$ is positive if the current is in the direction
$\sqrt{g} \nabla\zeta$, which is opposite to the $\nabla \zeta$ direction if $\sqrt{g} < 0$. 
In my understanding, $\sqrt{g}$ is \emph{always} $<0$ in
\vmec~output.
(The notation $\sqrt{g}$ is confusing, since square roots are normally defined to be positive.)
Thus, 
\begin{equation}
\mathtt{curtor} = \mathtt{signgs} \;I(1),
\label{eq:curtor}
\end{equation}
where {\ttfamily signgs}$=- 1$ is the 
name for the sign of $\sqrt{g}$ in the \vmec~{\ttfamily wout*.nc} output file.
In (\ref{eq:curtor}) we have used the \vmec~convention that $s=1$ is the
outermost magnetic surface in the code.
The particular definition of $s$ used in \vmec~is that $s$ is the toroidal flux
normalized to range from 0 on the magnetic axis to 1 at the outermost surface.


The other relevant \vmec~input parameters are \ac~or {\ttfamily ac\_aux\_f}; I'll write just \ac~here 
to denote whichever one is used based on the {\ttfamily pcurr\_type} parameter. The profile determined by \ac~(using a power series, spline, or other function) corresponds to 
$dI/ds$ up to an overall scale factor.
The \ac~profile is always scaled by \vmec~so the total current at the outermost surface is \curtor.
Hence,
\begin{equation}
\frac{dI/ds}{(dI/ds)_{s=1}}
=\frac{\mathtt{ac}(s)}{\mathtt{ac}(1)}.
\end{equation}
We can therefore set \ac~equal to (\ref{eq:iteration_low_beta}), to (\ref{eq:iteration_lag}), or to the radial derivative
of (\ref{eq:main_result}) (times any constant.)

To evaluate any of these three expressions, we need $d\psi/ds$.
Since $s\propto \psi$ in \vmec,
$d\psi/ds$ is just the constant $\psi_{edge}$,
the toroidal flux at the outermost magnetic surface.
However, we must be careful about the sign.
Applying the operation
$ \int_0^1 ds \int_0^{2\pi} d\theta \sqrt{g} \nabla\zeta \cdot (\ldots)$
to (\ref{eq:straight_field_lines}) gives
\begin{equation}
\int_0^1 ds \int_0^{2\pi} d\theta \sqrt{g} \vect{B} \cdot \nabla\zeta
= 
\int_0^1 ds \int_0^{2\pi} d\theta \sqrt{g} 
\frac{d\psi}{ds}  \nabla s \times\nabla \theta \cdot \nabla\zeta
=2 \pi \frac{d\psi}{ds} = 2\pi \psi_{edge}.
\end{equation}
From this equation, we see the sign convention for $d\psi/ds$ in (\ref{eq:main_result})
is that a positive value corresponds to the magnetic field pointing in the direction
$\sqrt{g} \nabla\zeta$.
In contrast, I believe \vmec's convention is that the output toroidal flux variable {\ttfamily phi}
is positive if the magnetic field points in the direction
$\nabla\zeta$. Also, {\ttfamily phi} is the flux \emph{not} divided by $2\pi$, in contrast to $\psi$ in this note.
Hence, the value of $d\psi/ds$ to use in (\ref{eq:main_result}) is
\begin{equation}
\frac{d\psi}{ds} = \mathtt{signgs}\frac{\mathtt{phi}(s=1)}{2\pi}.
\end{equation}
Note that the output quantity {\ttfamily phips} differs from {\ttfamily phi}
by a factor $2\pi${\ttfamily signgs}, so we could equivalently use $d\psi/ds = \mathtt{phips}$.


%\bibliographystyle{plain}
%\bibliography{computing_vmec_ac_profile_from_a_bootstrap_current_code}

\begin{thebibliography}{1}

\bibitem{LandremanCatto}
M~Landreman and P~J Catto.
\newblock {\em Phys. Plasmas}, {\bf 19}, 056103 (2012).

\end{thebibliography}

\end{document} 
