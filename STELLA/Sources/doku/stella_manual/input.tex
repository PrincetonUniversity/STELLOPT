\chapter{Inputs}\label{CHAP_INPUT}


\section{Geometry input: the \texttt{wout*.nc} file}

\section{The physics case input: the \texttt{*.in} file}

\stella~uses for its execution a single input file. In this section, 
we describe how this is structured in different 
namelists and which input variables can be included in each of these. The input file, 
which should include the suffix \texttt{.in} at the end of its name, is structured
if the following namelists:

\begin{itemize}
\item \ttt{zgrid\_parameters}
\item \ttt{geo\_knobs}
\item ...
\end{itemize}

In the following sections the description of the different input variables that 
each namelist has is provided. The purpose, description, variable type, etc
of each variable is displayed in tabular format, where the header shows the name of the 
variable in  teletype font family with the correponding mathematial mode symbol 
used in other parts of the documents, if any.


%%%%%%%%%%%%%%%%%%%%%%%%%%%%%%%%%%%%%%%%%%%%%%%%%%%%%%%%%%%%%%%%%%%%%%%%%%%%%%%%%%%%%%%%%%

\subsection{The {\ttfamily zgrid\_parameters} namelist}\label{sec:NL_ZGRID}


\myvar{\ttt{nzed} $(N_{\zeta})$}
{Variable for the grid of the spatial grid domain along $\zeta$.
It sets the number of divisions of the magnetic field line 
for the chosen flux tube along the coordinate $\zeta$.}
{integer}{}

\myvar{\ttt{nfield\_periods}}
{It defines the flux tube length along $\zeta$. 
  Being $\Nfp$ the number of field periods per $2\pi$ toroidal 
  segment of a given equilibrum (e.g. $\Nfp=5$ for W7-X), 
  \ttt{nfield\_periods} is the 
  number of toroidal field periods the flux tube extents along 
  the $\zeta$ direction. Considering the safety factor $q$ at 
  a given flux surface, set through the variable \ttt{torflux}, 
  if the flux tube is wanted to extent along $N_{\theta}$ poloidal 
  turns, i.e. covering the range $\left(-N_{\theta}\pi, N_{\theta}\pi\right)$ 
  in $\theta$, the following rule can be written to set accordingly 
  \ttt{nfield\_periods}number of divisions of the magnetic field line 
  for the chosen flux tube along the coordinate $\zeta$.
  \begin{equation}
    \ttt{nfield\_periods}=q\Nfp N_{\theta}
  \end{equation}}
{float}{}

\subsection{The \ttt{zgrid\_parameters} namelist}\label{sec:NL_ZGRID}
\subsection{The \ttt{geo\_knobs} namelist}
\subsection{The \ttt{vmec\_parameters} namelist}
\subsection{The \ttt{parameters} namelist}
\subsection{The \ttt{vpamu\_grids\_parameters} namelist}
\subsection{The \ttt{dist\_fn\_knobs} namelist}

\myvarv{adiabatic\_option}
{Form of the adiabatic response (if a species is being modeled as adiabatic). 
  Ignored if there are electrons in the species list.}
{\begin{itemize}
  \item \ttt{no-field-line-average-term}: adiabatic species has $n=\varphi$. 
    Appropriate for single-species ETG simulations.
  \item \ttt{field-line-average-term}: adiabatic species has $n=\varphi-\langle\varphi\rangle$. 
    Appropriate for single-species ITG simulations.
  \item \ttt{iphi00=0}: same as \ttt{no-field-line-average-term}.
  \item \ttt{iphi00=1}: same as \ttt{no-field-line-average-term}.
  \item \ttt{iphi00=2}: same as \ttt{field-line-average-term}.
  \item \ttt{iphi00=3}: adiabatic species has $n=\varphi-\langle\tilde{\varphi}\rangle_y$.
    Incorrect implementation of \ttt{field-line-average-term}
  \end{itemize}
}
{string}{\ttt{no-field-line-average-term}}

\subsection{The \ttt{time\_advance\_knobs}  namelist}
\subsection{The \ttt{kt\_grids\_knobs}  namelist}
\subsection{The \ttt{kt\_grids\_range\_parameters}  namelist}
\subsection{The \ttt{init\_g\_knobs}  namelist}
\subsection{The \ttt{knobs}  namelist}
\subsection{The \ttt{species\_knobs} namelist}
\subsection{The \ttt{species\_parameters\_\#} namelist}
\subsection{The \ttt{stella\_diagnostics\_knobs} namelist}
\subsection{The \ttt{reinit\_knobs} namelist}
\subsection{The \ttt{layouts\_knobs} namelist}
\subsection{The \ttt{neoclassical\_input} namelist}
\subsection{The \ttt{sfincs\_input} namelist}