\documentclass[11pt,letter]{article}

\usepackage{latexsym, color, graphicx, comment}
\usepackage[top=1in,bottom=1in,left=1in,right=1in]{geometry}
\usepackage{amssymb}
\usepackage{amsmath}
\usepackage{hyperref}
\usepackage{empheq}
\hypersetup{colorlinks=true,linkcolor=blue}

\newcommand{\vect}[1]{\mbox{\boldmath $#1$}}
\newcommand{\gyrophase}{\varphi}
\newcommand{\energy}{\varepsilon}
\renewcommand{\Re}{\mathrm{Re}}
\renewcommand{\Im}{\mathrm{Im}}
\newcommand{\todo}[1]{\textcolor{red}{#1}}
%\newcommand{\kxfac}{\kappa_x}
\newcommand{\kxfac}{\mathtt{kxfac}}
\newcommand{\bmag}{\mathtt{bmag}}
\newcommand{\smz}{\mathtt{smz}}
\newcommand{\gdstwo}{\mathtt{gds2}}
\newcommand{\gdstwoone}{\mathtt{gds21}}
\newcommand{\gdstwotwo}{\mathtt{gds22}}
\newcommand{\gbdrift}{\mathtt{gbdrift}}
\newcommand{\gbdriftO}{\mathtt{gbdrift0}}
\newcommand{\cvdrift}{\mathtt{cvdrift}}
\newcommand{\cvdriftO}{\mathtt{cvdrift0}}
\newcommand{\fprim}{\mathtt{fprim}}
\newcommand{\tprim}{\mathtt{tprim}}
\newcommand{\jtwist}{\mathtt{jtwist}}
\newcommand{\gradpar}{\mathtt{gradpar}}
\newcommand{\delthet}{\mathtt{delthet}}
\newcommand{\codedt}{\mathtt{code\_dt}}


\title{Definitions for GS2 full-flux-surface stellarator geometry}
\author{Matt Landreman}

\begin{document}
\maketitle

In this note, we state the definitions used for geometric quantities in the full-flux-surface
version of GS2, and we detail how these quantities are computed from the information
in a VMEC output file. We also derive how the twist-and-shift parallel boundary condition
and box size quantization conditions, familiar from flux tube simulations,
are modified in a full-surface calculation.

\section{Review of GS2 geometry definitions}

\subsection{Definitions valid for any GS2 geometry}
First, we review the definitions used in the flux tube version of GS2,
as discussed in the note {\ttfamily gs2\_geometry\_definitions.pdf}.
The Clebsch representation of the magnetic field is
\begin{equation}
\vect{B} = \nabla\psi\times\nabla\alpha.
\label{eq:Clebsch}
\end{equation}
We take $\psi$ to represent a flux surface label, but do not (yet) assume that it is necessarily the poloidal or toroidal flux. The $\alpha$ coordinate is a field line label on the flux surface.
New coordinates $x=x(\psi)$ and $y=y(\alpha)$ are introduced that are scaled versions of $\psi$ and $\alpha$ 
with dimensions of length. In a flux tube simulation where $x$ and $y$ only vary on a length scale comparable to the gyroradius, then
\begin{align}
&x = \frac{dx}{d\psi} \left[ \psi - \psi_0 \right], \\
&y = \frac{dy}{d\alpha} \left[ \alpha - \alpha_0 \right] \nonumber,
\end{align}
where $dx/d\psi$ and $dy/d\alpha$ are constant within the flux tube,
and $\psi_0$ and $\alpha_0$
are the coordinates around which the flux tube is centered.

A reference magnetic field strength $B_{ref}$ and reference length $L_{ref}$ are introduced for normalization. 
Then the quantities needed to specify the geometry in GS2 are the following:
\begin{align}
\bmag & = \frac{B}{B_{ref}}, \label{eq:bmag} \\
\gradpar &= L_{ref} \nabla_{||} z, \\
\gdstwo &= |\nabla y |^2  = \left( \frac{dy}{d\alpha} \right)^2 |\nabla\alpha|^2 , \\
\gdstwoone &= \hat{s} \nabla x \cdot \nabla y = \hat{s} \frac{dx}{d\psi} \frac{dy}{d\alpha} \nabla\psi\cdot\nabla\alpha , \\
\gdstwotwo &= \hat{s}^2|\nabla x |^2  = \hat{s}^2 \left( \frac{dx}{d\psi} \right)^2 |\nabla\psi |^2. \\
\gbdrift &= \frac{2 B_{ref} L_{ref}}{B^3} \vect{B}\times\nabla B\cdot \nabla y, \\
\gbdriftO &= \hat{s} \frac{2 B_{ref} L_{ref}}{B^3} \vect{B}\times\nabla B\cdot \nabla x,  \\
\cvdrift &= \frac{2 B_{ref} L_{ref}}{B^2} \vect{B}\times\vect{\kappa} \cdot \nabla y,  \\
\cvdriftO &= \hat{s} \frac{2 B_{ref} L_{ref}}{B^2} \vect{B}\times\vect{\kappa} \cdot \nabla x, \label{eq:cvdrift0} \\
\kxfac & = B_{ref} \frac{dx}{d\psi} \frac{dy}{d\alpha}, \label{eq:kxfac} \\
\fprim &=
B_{ref} L_{ref}
\frac{dy}{d\alpha} \frac{dx}{d\psi} \frac{1}{n_s} \frac{d n_s}{d x}, \\
\tprim &=
B_{ref} L_{ref}
\frac{dy}{d\alpha} \frac{dx}{d\psi} \frac{1}{T_s} \frac{d T_s}{d x}. \label{eq:tprim}
\end{align}
These quantities are all dimensionless. In the definition of $\gradpar$, $z$ is any parallel coordinate; the parallel coordinate
is called {\ttfamily theta} in GS2, although it need not be the poloidal angle.
Also $\hat{s}$ is whichever number is used to define $\theta_0=${\ttfamily theta0} in the relation
\begin{equation}
\theta_0 = \frac{k_x}{\hat{s} k_y}.
\end{equation}
The above quantities (\ref{eq:bmag})-(\ref{eq:cvdrift0}) are functions of the parallel coordinate $z$.
The quantity $\kxfac$ is a single number, and the quantities $\fprim$ and $\tprim$  are each single numbers
for each particle species.

Since $\vect{B}\times\vect{\kappa}\cdot\nabla\psi = \vect{b}\times\nabla B\cdot\nabla\psi$ for any
static ideal MHD equilibrium (at any $\beta$), then $\cvdriftO=\gbdriftO$.

\subsection{Definitions used previously for stellarator geometry in GS2}

In stellarator calculations that have been performed using the standard flux tube version of GS2
and the GIST geometry interface,
the following additional definitions have been made. 
The flux surface label $\psi$ has been taken to be the toroidal flux divided by $2\pi$.
Therefore for consistency with (\ref{eq:Clebsch}),
$\alpha = \theta - \iota \zeta$ where $\iota = 1/q$ is the rotational transform,
$q$ is the safety factor, and $\theta$ and $\zeta$ are straight-field-line poloidal and toroidal angles.
While the GIST GS2 interface takes $\theta$ and $\zeta$ to be Boozer
angles, all the expressions below are equally valid for other straight-field-line angles such as PEST or Hamada coordinates.

In GIST, the parallel coordinate $z$ is presently taken to be the Boozer poloidal angle. However none of the expressions in this note are altered if a different choice is desired.

In GIST, the reference length $L_{ref}$ is taken to be the effective minor radius computed by VMEC, named Aminor\_p in
the VMEC wout*.nc file.
The reference magnetic field is taken to be
\begin{equation}
B_{ref} = \frac{2 \psi_{LCFS}}{L_{ref}^2}
\label{eq:Bref}
\end{equation}
where $\psi_{LCFS}$ is the value of $\psi$ at the outermost VMEC flux surface. The choice (\ref{eq:Bref}) is
motivated by the cylindrical limit: the toroidal flux enclosed by the outermost VMEC surface is equivalent to the flux of a field $B_{ref}$ through a circle of radius $L_{ref}$.



The radial coordinate $x$ is then chosen to be
\begin{equation}
x = L_{ref} \sqrt{\frac{\psi}{\psi_{LCFS}}} = L_{ref} \sqrt{s},
\label{eq:gist_x}
\end{equation}
where $s = \psi / \psi_{LCFS} \in [0,1]$ is the flux surface label coordinate
used in VMEC. 
The choice (\ref{eq:gist_x}) is natural since 
$x$ then reduces to the usual minor radius in the cylindrical limit.
%the toroidal flux $2\pi\psi$ scales like the square of the radius in the cylindrical limit.
It follows that
\begin{equation}
\frac{dx}{d\psi} = \frac{L_{ref}}{2 \sqrt{ \psi \psi_{LCFS}}}
= \frac{1}{L_{ref} B_{ref}} \sqrt{ \frac{\psi_{LCFS}}{\psi}}.
\label{eq:psi_x_conversion}
\end{equation}

Another choice made in GIST is $\kxfac=1$. According to (\ref{eq:kxfac}), we are then required to take 
\begin{equation}
\frac{d y}{d\alpha} = L_{ref} \sqrt{\frac{\psi}{\psi_{LCFS}}} .
\label{eq:dy_dalpha}
\end{equation}

GIST computes the global shear parameter $\hat{s}$ using
\begin{equation}
\hat{s} = \frac{x}{q} \frac{dq}{dx}.
\end{equation}

Now that $dx/d\psi$ and $dy/d\alpha$ are specified, equations (\ref{eq:bmag})-(\ref{eq:tprim}) 
can be evaluated to obtain explicit expressions for the geometry arrays computed by GIST for GS2:
\begin{eqnarray}
\bmag &=& B/B_{ref}, \label{eq:gist_Bref}\\
\gradpar &=& L_{ref} \nabla_{||} z, \\
\gdstwo &=& |\nabla y|^2 = |\nabla\alpha|^2 L_{ref}^2 \frac{\psi}{\psi_{LCFS}}, \\
\gdstwoone &=& \hat{s} \nabla x \cdot \nabla y = \frac{\hat{s}}{B_{ref}} \nabla\psi\cdot\nabla\alpha , \\
\gdstwotwo &=& \hat{s}^2|\nabla x |^2  = \left( \frac{\hat{s}}{L_{ref} B_{ref}}\right)^2 \frac{\psi_{LCFS}}{\psi} |\nabla\psi |^2, \\
\gbdrift &=& \frac{2 B_{ref} L_{ref}^2}{B^3} \sqrt{\frac{\psi}{\psi_{LCFS}}} \vect{B}\times\nabla B\cdot \nabla \alpha, \\
\gbdriftO &=& \hat{s} \frac{2 }{B^3}   \sqrt{\frac{\psi_{LCFS}}{\psi}} \vect{B}\times\nabla B\cdot \nabla \psi, \\
\cvdrift &=& \frac{2 B_{ref} L_{ref}^2}{B^2} \sqrt{\frac{\psi}{\psi_{LCFS}}} \vect{B}\times\vect{\kappa} \cdot \nabla \alpha, \\
\cvdriftO &=& \hat{s} \frac{2 }{B^2}  \sqrt{\frac{\psi_{LCFS}}{\psi}} \vect{B}\times\vect{\kappa} \cdot \nabla \psi,  \label{eq:gist_cvdrift0} \\
\fprim &=&   \frac{L_{ref}}{n_s} \frac{d n_s}{d x}, \\
\tprim &=&   \frac{L_{ref}}{T_s} \frac{d T_s}{d x}. 
\end{eqnarray}

\section{VMEC coordinates}

The VMEC code uses a toroidal coordinate $\zeta$ which is the conventional azimuthal angle of cylindrical coordinates. 

We let $\theta_v$ denote the poloidal angle used in VMEC, which is \emph{not} a straight-field-line coordinate.
We also let $\theta_p$ denote the PEST poloidal angle, i.e. the straight-field-line angle which results when
the toroidal angle is chosen to be the conventional azimuthal angle of cylindrical coordinates, as in VMEC.
The conversion between the two coordinates is
\begin{equation}
\theta_p = \theta_v + \Lambda,
\end{equation}
where $\Lambda$ is the quantity given by the {\ttfamily lmns} and {\ttfamily lmnc} arrays in VMEC.
Thus, the field line label we need for GS2 geometry quantities is
\begin{equation}
\alpha = \theta_v + \Lambda - \iota \zeta.
\label{eq:alpha_vmec}
\end{equation}

VMEC provides many quantities as functions of the coordinates $(s, \theta_v, \zeta)$,
where again $s = \psi / \psi_{LCFS}$.
Specifically, it provides the Fourier amplitudes for expansions in $\theta_v$ and $\zeta$,
on grid points equally spaced in $s$.
The quantities that are available include
$\Lambda$, $B$, the cylindrical coordinates $(R,Z)$, the components
\begin{align}
B^\theta &= \vect{B}\cdot\nabla\theta_v, \nonumber \\
B^\zeta &= \vect{B}\cdot\nabla\zeta, \nonumber \\
B_s &= \vect{B}\cdot\frac{\partial\vect{r}}{\partial s}, \nonumber \\
B_\theta &= \vect{B}\cdot\frac{\partial\vect{r}}{\partial \theta_v}, \nonumber \\
B_\zeta &= \vect{B}\cdot\frac{\partial\vect{r}}{\partial \zeta}, \nonumber
\end{align}
and the Jacobian
\begin{equation}
\sqrt{g} = \frac{\partial \vect{r}}{\partial s} \cdot \frac{\partial \vect{r}}{\partial\theta_v} \times \frac{\partial\vect{r}}{\partial\zeta}
= \frac{1}{\nabla s \cdot \nabla \theta_v \times\nabla\zeta}.
\label{eq:Jacobian}
\end{equation}
Here $\vect{r}(s,\theta_v,\zeta)$ is the position vector.
Throughout this note, we will use $\sqrt{g}$ to denote the Jacobian of the non-straight-field-line
VMEC coordinates, as in (\ref{eq:Jacobian}).

%\begin{align}
%\vect{B} &= \nabla\psi \times \nabla\theta_p + \iota \nabla\zeta \times \nabla\psi \nonumber \\
%& = \nabla\psi \times \nabla\theta_v + \nabla\psi \times \nabla \Lambda + \iota \nabla\zeta \times \nabla\psi .
%\end{align}

\section{Computation of GS2 geometry quantities from VMEC data} 


Given the desired central flux surface $\psi_0$, and given a desired set of grid points in $\alpha$ and $\zeta$, 
a 1D nonlinear root-finding algorithm is applied to solve (\ref{eq:alpha_vmec}) for the 
value of $\theta_v$ at each grid point. The $\bmag$ array is then obtained by evaluating $B$
at the $(\theta_v, \zeta)$ grid points, using VMEC's Fourier arrays {\ttfamily bmnc} and {\ttfamily bmns}.

For the full-flux-surface calculation, we will take the parallel coordinate $z$ to be the toroidal angle $\zeta$.
Then to evaluate $\gradpar$, we use
\begin{equation}
\gradpar = L_{ref} \frac{\vect{B}\cdot\nabla\zeta}{B} = L_{ref} \frac{B^\zeta}{B},
\end{equation}
%\begin{align}
%\gradpar
%&=\frac{\vect{B}\cdot\nabla\zeta}{B}
%=\frac{\nabla\psi\times\nabla\alpha\cdot\nabla\zeta}{B}
%=\frac{\nabla\psi\times\nabla(\theta_v +\Lambda - \iota\zeta)\cdot\nabla\zeta}{B}
%=\frac{\nabla\psi\times\nabla(\theta_v +\Lambda)\cdot\nabla\zeta}{B} \nonumber \\
%&=\frac{1}{B \sqrt{g}} \left( 1 + \frac{\partial\Lambda}{\partial\theta_v}\right).
%\end{align}
where $B^\zeta$ is available in the VMEC output through the variables {\ttfamily bsupvmnc} and {\ttfamily bsupvmns}.

To evaluate the quantities {\ttfamily gds*}, we must obtain the Cartesian components of
$\nabla\psi$ and $\nabla \alpha$. To this end, we use the dual relations:
\begin{align}
\nabla s &=  \frac{1}{\sqrt{g}} \frac{\partial\vect{r}}{\partial\theta_v} \times \frac{\partial\vect{r}}{\partial\zeta}, \label{eq:nabla_s}\\
\nabla\theta_v &= \frac{1}{\sqrt{g}} \frac{\partial\vect{r}}{\partial\zeta} \times \frac{\partial\vect{r}}{\partial s}
\label{eq:nabla_theta}.
\end{align}
The right hand sides of these two expressions can be evaluated in terms of Cartesian components using the VMEC
outputs {\ttfamily rmnc}, {\ttfamily rmns}, {\ttfamily zmnc}, and {\ttfamily zmns}, yielding
Cartesian components for $\nabla s$ and $\nabla\theta_v$. Also the Cartesian components of $\nabla\zeta$
are known since $\zeta$ is the standard toroidal angle.
We can then compute
\begin{equation}
\nabla\psi = \frac{d\psi}{ds} \nabla s = \psi_{LCFS}  \frac{1}{\sqrt{g}} \frac{\partial\vect{r}}{\partial\theta_v} \times \frac{\partial\vect{r}}{\partial\zeta}
\end{equation}
and
\begin{align}
\nabla\alpha 
&= \nabla(\theta_v + \Lambda - \iota \zeta) \nonumber \\
&= \left(\frac{\partial\Lambda}{\partial s} - \zeta \frac{d\iota}{ds}\right)\nabla s
+ \left( 1 + \frac{\partial\Lambda}{\partial\theta_v}\right) \nabla\theta_v
+ \left( -\iota + \frac{\partial\Lambda}{\partial\zeta}\right) \nabla\zeta.
\label{eq:grad_alpha}
\end{align}
\todo{Do we want to subtract {\ttfamily zeta\_center} from $\zeta$ here so the secular $\zeta$ term in $\nabla\alpha$ vanishes at the center of the domain? Or might we want it to vanish somehwhere other than the center of the domain?}
Now that Cartesian components of $\nabla\psi$ and $\nabla \alpha$ are known, $\gdstwo$, $\gdstwoone$, and $\gdstwotwo$ can be computed. 

To evaluate $\gbdriftO = \cvdriftO$, we can use
\begin{align}
\vect{B}\times\nabla B\cdot\nabla\psi 
&= 
\vect{B}\times\nabla \zeta \cdot \nabla \psi \frac{\partial B}{\partial \zeta} + \vect{B}\times\nabla \theta_v \cdot \nabla \psi \frac{\partial B}{\partial \theta_v} \nonumber \\
&=  \left(
B_\theta \nabla\theta_v \times\nabla\zeta\cdot\nabla s \frac{\partial B}{\partial\zeta} + B_\zeta \nabla\zeta\times\nabla\theta_v\cdot\nabla s \frac{\partial B}{\partial\theta_v} \right) \frac{d\psi}{ds} \nonumber \\
&= \left( B_\theta \frac{\partial B}{\partial\zeta} - B_\zeta \frac{\partial B}{\partial\theta_v} \right) \frac{\psi_{LCFS} }{\sqrt{g} }.
\end{align}
To obtain this result we have used
\begin{equation}
\nabla B = \frac{\partial B}{\partial s} \nabla s + \frac{\partial B}{\partial\theta_v} \nabla\theta_v + \frac{\partial B}{\partial\zeta} \nabla\zeta
\label{eq:gradB}
\end{equation}
and
\begin{equation}
\vect{B} = B_s \nabla s + B_\theta \nabla \theta_v + B_\zeta \nabla\zeta.
\label{eq:Bsub}
\end{equation}
The quantity $B_\theta$ is available as the VMEC outputs {\ttfamily bsubumnc} and {\ttfamily bsubumns},
and the quantity $B_\zeta$ is available as the VMEC outputs {\ttfamily bsubvmnc} and {\ttfamily bsubvmns}.


A couple of options are possible for computing $\gbdrift$. One method is to compute the Cartesian components of $\nabla B$ using
(\ref{eq:gradB})
together with (\ref{eq:nabla_s})-(\ref{eq:nabla_theta}). Furthermore, the Cartesian components of $\vect{B}$ can be computed from
\begin{align}
\vect{B} &= \nabla\psi \times \nabla (\theta_v + \Lambda) + \iota \nabla\zeta\times\nabla\psi \nonumber \\
&= \frac{d\psi}{ds} \left[ \left( 1 + \frac{\partial\Lambda}{\partial\theta_v}\right) \nabla s \times\nabla\theta_v 
+ \left(\iota - \frac{\partial\Lambda}{\partial\zeta}\right) \nabla\zeta\times\nabla s \right] \nonumber \\
&= \frac{\psi_{LCFS}}{\sqrt{g}} \left[ \left(1 + \frac{\partial\Lambda}{\partial\theta_v}\right) \frac{\partial\vect{r}}{\partial\zeta}
+ \left( \iota - \frac{\partial\Lambda}{\partial\zeta}\right) \frac{\partial\vect{r}}{\partial\theta_v}\right].
\end{align}
Now that Cartesian components of $\vect{B}$, $\nabla B$, and $\nabla \alpha$ are all known, their cross product
needed for $\gbdrift$ is straightforward. Alternatively, $\gbdrift$ can be computed by substituting (\ref{eq:gradB})-(\ref{eq:Bsub}) and (\ref{eq:grad_alpha}) into $\vect{B}\times\nabla B \cdot\nabla\alpha$, yielding
\begin{align}
\vect{B}\times\nabla B \cdot\nabla\alpha
=
\frac{1}{\sqrt{g}} &\left[
B_s \frac{\partial B}{\partial \theta_v} \left(\frac{\partial \Lambda}{\partial\zeta} - \iota \right)
+ B_\theta \frac{\partial B}{\partial\zeta} \left( \frac{\partial\Lambda}{\partial s} - \zeta \frac{d\iota}{ds}\right)
+ B_\zeta \frac{\partial B}{\partial s} \left( 1 + \frac{\partial \Lambda}{\partial \theta_v}\right) \right. \nonumber \\
&\left.-B_\zeta \frac{\partial B}{\partial\theta_v} \left( \frac{\partial\Lambda}{\partial s} - \zeta \frac{d\iota}{ds}\right)
-B_\theta \frac{\partial B}{\partial s} \left( \frac{\partial \Lambda}{\partial\zeta}-\iota\right)
-B_s \frac{\partial B}{\partial\zeta} \left( 1 + \frac{\partial\Lambda}{\partial\theta_v}\right)
\right].
\end{align}


The last quantity we need to evaluate is $\cvdrift$. Using
\begin{align}
\vect{B}\times\vect{\kappa}
&= \vect{B}\times(\vect{b}\cdot\nabla \vect{b}) 
= \vect{B}\times \left[ (\nabla\times\vect{b}) \times\vect{b}\right] \nonumber \\
&= \vect{B} \times \left[ -\frac{1}{B^2} \left( \nabla B \times\vect{B}\right)\times\vect{b} + \frac{\mu_0}{B^2} \vect{j}\times\vect{B}\right] \nonumber \\
&= \frac{1}{B} \vect{B}\times\nabla B + \frac{\mu_0}{B^2} \frac{dp}{ds} \vect{B}\times\nabla s,
\end{align}
(where we have used the MHD equilibrium equation $\vect{j}\times\vect{B} = \nabla p$), we find
\begin{equation}
\cvdrift = \gbdrift 
+ \frac{2 B_{ref} L_{ref}^2}{B^2} \sqrt{\frac{\psi}{\psi_{LCFS}}} \frac{\mu_0}{B^2} \frac{dp}{ds} \vect{B}\times\nabla s \cdot \nabla \alpha.
\end{equation}
In the last term, $\vect{B}\times\nabla s \cdot \nabla \alpha$ can either
be evaluated using the Cartesian components of $\vect{B}$, $\nabla s$, and $\nabla \alpha$,
the calculation of which has already been described,
or by combining (\ref{eq:grad_alpha}) and (\ref{eq:Bsub}) to obtain
\begin{equation}
\vect{B}\times\nabla s \cdot \nabla \alpha
= \frac{1}{\sqrt{g}} \left[
B_\zeta \left( 1  + \frac{\partial\Lambda}{\partial\theta_v}\right)
-B_\theta \left( \frac{\partial\Lambda}{\partial\zeta} - \iota \right) \right].
\end{equation}




\section{Parallel boundary condition and wavenumber quantization for a full surface calculation}

Let us now consider a full-flux-surface calculations, using field-aligned coordinates. 
We continue to use $\psi$ and $\alpha$ as perpendicular coordinates, and $\zeta$ as the third (parallel) coordinate.
In a full-flux-surface calculation, the range of $\alpha$ is $[0,\; 2\pi)$,
which can be seen from the fact at fixed $\psi$ and $\zeta$, the $2\pi$-periodicity of quantities in
$\theta_p$ implies $2\pi$-periodicity in $\alpha$.

Just as in a flux tube code, fluctuating quantities such as the electrostatic potential $\phi$ are represented as
\begin{equation}
\phi(\psi,\alpha,\zeta)
=\sum_{k_\psi, k_\alpha} \bar{\phi}_{k_\psi,k_\alpha}(\zeta) \exp\left( i k_\alpha \alpha + i k_\psi \left[ \psi - \psi_0\right] \right).
\label{eq:fluctuations}
\end{equation}
Again, $\psi_0$ indicates the flux surface about which the numerical domain is centered.
The wavenumbers with respect to $(\psi,\alpha)$ are related to wavenumbers with respect to $(x,y)$ through
\begin{align}
\label{eq:k_conversion}
k_x &= k_\psi \frac{d\psi}{dx}, \\
k_y &= k_\alpha \frac{d\alpha}{dy} \nonumber.
\end{align}
Note that $k_\alpha$ ranges over the integers, due to the $2\pi$-periodicity in $\alpha$ discussed above.
The choice of $y$ (\ref{eq:dy_dalpha}) then implies the wavenumber grid in $y$ must be
\begin{equation}
k_y \rho_{ref} = \frac{\rho_{ref}}{L_{ref} \sqrt{s(\psi_0)}} \times (\mathrm{integers}).
\end{equation}
We take fluctuating quantities to be periodic in $\psi$, just as in a flux tube code. The allowed values of $k_\psi$ and associated `box size' in $\psi$
will be derived below.

We take fluctuating quantities to be periodic in $\zeta$ with period $2 \pi P$ at fixed $\psi$ and $\theta_p$.
Here, $P$ is a rational number, typically the inverse of the number of field periods (e.g. 5 for W7-X).
One could also choose $P=1$ to simulate the entire toroidal domain, or choose $P=$ integer / (number of field periods)
for an intermediate domain size. When $P=1$, this periodicity condition is the true periodicity of the torus.
When $P=$ integer / (number of field periods), the periodicity imposed in the code is effectively a statement
of statistical periodicity of the turbulence at geometrically equivalent points in the domain. To see the implications of imposing
periodicity in $\zeta$ at fixed $\theta_p$, we substitute $\alpha = \theta_p - \iota \zeta$ and the Taylor expansion
\begin{equation}
\iota \approx \iota(\psi_0) + \frac{d\iota}{d\psi} \left[ \psi - \psi_0\right]
\label{eq:Taylor_iota}
\end{equation}
into (\ref{eq:fluctuations}). (We take $d\iota/d\psi$ to be evaluated at $\psi_0$, and hence constant over the domain.) The result is
\begin{equation}
\phi
=\sum_{k_\psi, k_\alpha} \bar{\phi}_{k_\psi,k_\alpha}(\zeta) \exp
\left( i k_\alpha \theta_p  
-i k_\alpha \iota(\psi_0) \zeta
+i \left[- k_\alpha \frac{d\iota}{d\psi}  \zeta
 + k_\psi\right] \left[ \psi - \psi_0\right] \right).
\end{equation}
When a particular $(k_\psi,k_\alpha)$ Fourier mode gets to the end of the $\zeta$ domain ($\zeta = 2\pi P$), we want
the mode to connect exactly to another Fourier mode in the simulation $(k'_\psi,k'_\alpha)$, with the latter evaluated
at $\zeta=0$. Mathematically, this condition is
\begin{align}
&\bar{\phi}_{k_\psi,k_\alpha}(2\pi P) \exp
\left( i k_\alpha \theta_p  
-i k_\alpha \iota(\psi_0) 2\pi P
+i \left[- k_\alpha \frac{d\iota}{d\psi}  2 \pi P
+ k_\psi\right] \left[ \psi - \psi_0\right] \right) \nonumber \\
&=
\bar{\phi}_{k'_\psi,k'_\alpha}(0) \exp
\left( i k'_\alpha \theta_p  
+i k'_\psi\left[ \psi - \psi_0\right] \right) .
\label{eq:parallelBC}
\end{align}
For this equation to hold for all $\theta_p$ at fixed $\psi$, we must have $k'_\alpha = k_\alpha$.
For equality to hold for all $\psi$ at fixed $\theta_p$, we must have
\begin{equation}
k'_\psi
= k_\psi - k_\alpha \frac{d\iota}{d\psi}  2 \pi P.
\label{eq:twist_and_shift}
\end{equation}
This last result is analogous to the twist-and-shift condition used in flux-tube GS2. 
If $k'_\psi$ is to be included in the wavenumber grid, and assuming the $k_\psi$ grid consists of integer multiples of some
$k_{\psi,min}$, then the difference $k'_\psi - k_\psi$ should
be an integer multiple of $k_{\psi,min}$. 
In particular this must be true for the smallest nonzero $k_\alpha$, which is 1.
Thus, we conclude
\begin{equation}
\frac{d\iota}{d\psi} 2\pi P = (\jtwist) k_{\psi,min}
\end{equation}
where $\jtwist$ is an integer. It follows that the `box size' in $\psi$, denoted $L_\psi$, is
\begin{equation}
L_\psi = \frac{2\pi}{k_{\psi,min}} = 
\frac{(\jtwist)}{P} \left( \frac{d\iota}{d\psi} \right)^{-1}.
\end{equation}
Using (\ref{eq:k_conversion}) and (\ref{eq:psi_x_conversion}), the equivalent box size in $x$ is
\begin{equation}
L_x = \frac{dx}{d\psi} L_\psi = \frac{2\pi}{k_{x,min}} = 
\frac{(\jtwist)}{ P L_{ref} B_{ref} \sqrt{s}} \left( \frac{d\iota}{d\psi} \right)^{-1}.
\label{eq:Lx}
\end{equation}
Unfortunately, in low-shear stellarators like W7-X and HSX, $d \iota/d\psi$ in 
(\ref{eq:Lx}) can be quite small, meaning $L_x$ must be quite large.

Equation (\ref{eq:parallelBC}) also indicates there should be a phase shift when two Fourier modes
are connected:
\begin{equation}
\bar{\phi}_{k_\psi,k_\alpha}(2\pi P) \exp
\left( -i k_\alpha \iota(\psi_0) 2\pi P \right) 
=
\bar{\phi}_{k'_\psi,k_\alpha}(0).
\label{eq:phase}
\end{equation}

\subsection{Continuity of geometric quantities}

%It is interesting to check whether 
We can verify that various terms in the gyrokinetic equation are continuous
% when a Fourier mode is followed 
across the parallel boundary condition. First, let us consider the magnetic drift term:
\begin{align}
\vect{v}_{m}\cdot \nabla h 
&= \vect{v}_{m} \cdot \nabla \left[
\sum_{k_\psi,k_\alpha} \bar{h}_{k_\psi,k_\alpha}(\zeta) \exp \left( i k_\alpha \alpha + i k_\psi [\psi - \psi_0] \right) \right] \nonumber \\
&\approx 
\sum_{k_\psi,k_\alpha} \bar{h}_{k_\psi,k_\alpha}(\zeta)
\exp \left( i k_\alpha \alpha + i k_\psi [\psi - \psi_0] \right)
\left[ i k_\alpha \vect{v}_{m}\cdot \nabla \alpha + i k_\psi \vect{v}_{m}\cdot\nabla\psi \right] \nonumber \\
& =
\sum_{k_\psi,k_\alpha} \bar{h}_{k_\psi,k_\alpha}(\zeta)
\exp \left( i k_\alpha [\theta_p - \iota \zeta] + i k_\psi [\psi - \psi_0] \right) \nonumber \\
& \hspace{1in}\times \left[ i k_\alpha \vect{v}_{m}\cdot \left(\nabla \theta_p - \iota \nabla\zeta - \zeta \frac{d\iota}{d\psi} \nabla\psi \right) + i k_\psi \vect{v}_{m}\cdot\nabla\psi \right],
\label{eq:grad_B_drift_term}
\end{align}
where $h$ is the nonadiabatic distribution function. (The $\approx$ above comes from dropping the slow $\zeta$ dependence of $\bar{h}_{k_\psi,\zeta}$
in the gradient.) As $\zeta$ is increased, just before the boundary $\zeta = 2\pi P$ we have
\begin{align}
\left( \vect{v}_{m}\cdot \nabla h \right)_-
& =
\sum_{k_\psi,k_\alpha} \bar{h}_{k_\psi,k_\alpha}(2\pi P)
\exp \left( i k_\alpha [\theta_p - \iota 2\pi P] + i k_\psi [\psi - \psi_0] \right) \nonumber \\
& \hspace{1in}\times \left[ i k_\alpha \vect{v}_{m}\cdot \left(\nabla \theta_p - \iota \nabla\zeta - 2\pi P \frac{d\iota}{d\psi} \nabla\psi \right) + i k_\psi \vect{v}_{m}\cdot\nabla\psi \right].
\label{eq:v_B_term_minus}
\end{align}
The subscript on the left hand side indicates that we have evaluated the result just to the left of the boundary.
On the other side of the boundary ($\zeta = 0$), (\ref{eq:grad_B_drift_term}) evaluates to
\begin{equation}
\left(\vect{v}_{m}\cdot \nabla h \right)_+
= 
\sum_{k_\psi,k_\alpha} \bar{h}_{k'_\psi,k_\alpha}(0)
\exp \left( i k_\alpha \theta_p  + i k'_\psi [\psi - \psi_0] \right) 
\left[ i k_\alpha \vect{v}_{m}\cdot \left(\nabla \theta_p - \iota \nabla\zeta \right) + i k'_\psi \vect{v}_{m}\cdot\nabla\psi \right].
\end{equation}
In this last equation, we were free to put primes on $k_\psi$ because it is summed over. Applying 
(\ref{eq:twist_and_shift}) and (\ref{eq:phase}),
\begin{align}
\left(\vect{v}_{m}\cdot \nabla h \right)_+
=& 
\sum_{k_\psi,k_\alpha} \bar{h}_{k_\psi,k_\alpha}(2\pi P) 
\exp \left( i k_\alpha \theta_p   -i k_\alpha \iota(\psi_0) 2\pi P+ i \left[ k_\psi - k_\alpha \frac{d\iota}{d\psi}  2 \pi P\right] [\psi - \psi_0] \right) \nonumber \\
& \hspace{1in} \times \left[ i k_\alpha \vect{v}_{m}\cdot \left(\nabla \theta_p - \iota \nabla\zeta \right) + i \left[ k_\psi - k_\alpha \frac{d\iota}{d\psi}  2 \pi P\right] \vect{v}_{m}\cdot\nabla\psi \right].
\label{eq:v_B_term_plus}
\end{align}
Using (\ref{eq:Taylor_iota}), it can be seen that (\ref{eq:v_B_term_minus}) is identical to  (\ref{eq:v_B_term_plus}),
so the magnetic drift term is indeed continuous across the boundary.

\todo{Also show continuity for $\langle \phi \rangle_R$...}

\begin{comment}

perpendicular wavenumber appearing inside the Bessel functions:
\begin{equation}
k_{\perp}^2 = k_\alpha^2 |\nabla \alpha|^2 + 2 k_\alpha k_\psi \nabla\alpha\times\nabla\psi + k_\psi^2 |\nabla\psi|^2.
\label{eq:kperp2}
\end{equation}
Using
\begin{equation}
\nabla\alpha = \nabla\theta_p - \iota \nabla\zeta - \zeta \frac{d\iota}{d\psi} \nabla\psi,
\end{equation}
we find that (\ref{eq:kperp2}) 

\end{comment}

\end{document} 
